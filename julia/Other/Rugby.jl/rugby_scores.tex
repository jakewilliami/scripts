% !TEX TS-program = pdflatexmk.py
\documentclass{arteacle}
%use the option bib-apa to use apa-style referencing

% To change any options that are being overwritten by the class,
% You will have to put this in an \AtEndPreamble statement

\title{What is a ``valid'' score in the game of rugby?}
\author{Jake W. Ireland}

\begin{document}

\subsubsection*{Context}

There are three ways you can get points in the game of rugby:
\begin{enumerate}
	\item A ``try'' is worth five points;
	\item A ``conversion'' is worth two points, but must be scored after a try;
	\item A non-conversion kick (often called, but not limited to, a ``penalty'') is worth three points.
\end{enumerate}

Let the set of individual scores be $S$, where
\begin{equation}
	S = \{2, 3, 5\}.
\end{equation}

Let us also say that the set of all valid scores in rugby $S^\prime$ is exactly $\{\naturals\union 0\}$ (i.e., no negative numbers).

\prop{
We want to prove that the only scores that are impossible to obtain in this game are scores of 1, 2, and 4.
}

\subsubsection*{Zero}

We start with the base case: you can get zero points, as everyone starts with zero points at the start of the game.  The lack of scoring anything will leave you with zero points.

\subsubsection*{One}

We can also not score 1 because
\begin{equation}
	1 < s\;\forall s\in S.
\end{equation}

\subsubsection*{Two}

We cannot obtain a score of 2, because we can only score a conversion under the condition that we score a try first.

\subsubsection*{Three}

Scoring a 3 is as ``simple'' as kicking the ball over the H-looking thing at the correct end of the field.

\subsubsection*{Four}

The only permutation of $S$ that makes is two of the score 2.  By the same logic as the score 2, we cannot obtain this because we can only score a conversion by first scoring a try.

\subsubsection*{Five}

We can score 5 points by \emph{committing} a try.

\subsubsection*{Six}

The score 6 can be obtained by scoring 3 twice.

\subsubsection*{Seven}

Score a try, and successfully convert the ball over the H thing.

\subsubsection*{Induction Step}

By induction, we can score 8 by scoring 5 (defined above) and scoring another 3 (also defined above).  We can score 9 by scoring 6 and scoring another 3.  We can score 10 by scoring 7 and scoring another 3.  

In general, we can write a recursive function $s$ of the score $n$ such that:
%\begin{equation}
%	s(n) = \begin{cases}
%		s(n - 3) + 3&\text{ where }n>7\\
%		5 + 2&\text{ where }n=7\\
%		3+3&\text{ where }n=6\\
%		5&\text{ where }n=5\\
%		3&\text{ where }n=3
%	\end{cases}
%\end{equation}
\begin{equation}
	s(n) = \begin{cases}
		s(n - 3) + 3,&\text{if }n>7\\
		a+b,&\text{if }n\leq 7\text{, with }a,b\in\{S\union 0\}
	\end{cases}.
\end{equation}

Hence, the only scores we cannot obtain are 1, 2, and 4, and by induction every other score is possible:
\begin{equation}
i \notin S^\prime \;\forall i \in \{1, 2, 4\}.
\end{equation}

\end{document}
